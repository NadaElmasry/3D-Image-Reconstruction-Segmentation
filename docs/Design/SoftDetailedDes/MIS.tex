\documentclass[12pt, titlepage]{article}

\usepackage{amsmath, mathtools}

\usepackage[round]{natbib}
\usepackage{amsfonts}
\usepackage{amssymb}
\usepackage{graphicx}
\usepackage{colortbl}
\usepackage{xr}
\usepackage{hyperref}
\usepackage{longtable}
\usepackage{xfrac}
\usepackage{tabularx}
\usepackage{float}
\usepackage{siunitx}
\usepackage{booktabs}
\usepackage{multirow}
\usepackage[section]{placeins}
\usepackage{caption}
\usepackage{fullpage}

\hypersetup{
bookmarks=true,     % show bookmarks bar?
colorlinks=true,       % false: boxed links; true: colored links
linkcolor=red,          % color of internal links (change box color with linkbordercolor)
citecolor=blue,      % color of links to bibliography
filecolor=magenta,  % color of file links
urlcolor=cyan          % color of external links
}

\usepackage{array}

\externaldocument{../../SRS/SRS}

%% Comments

\usepackage{color}

\newif\ifcomments\commentstrue %displays comments
%\newif\ifcomments\commentsfalse %so that comments do not display

\ifcomments
\newcommand{\authornote}[3]{\textcolor{#1}{[#3 ---#2]}}
\newcommand{\todo}[1]{\textcolor{red}{[TODO: #1]}}
\else
\newcommand{\authornote}[3]{}
\newcommand{\todo}[1]{}
\fi

\newcommand{\wss}[1]{\authornote{blue}{SS}{#1}} 
\newcommand{\plt}[1]{\authornote{magenta}{TPLT}{#1}} %For explanation of the template
\newcommand{\an}[1]{\authornote{cyan}{Author}{#1}}

%% Common Parts

\newcommand{\progname}{ProgName} % PUT YOUR PROGRAM NAME HERE
\newcommand{\authname}{Team \#, Team Name
\\ Student 1 name
\\ Student 2 name
\\ Student 3 name
\\ Student 4 name} % AUTHOR NAMES                  

\usepackage{hyperref}
    \hypersetup{colorlinks=true, linkcolor=blue, citecolor=blue, filecolor=blue,
                urlcolor=blue, unicode=false}
    \urlstyle{same}
                                


\begin{document}

\title{Module Interface Specification for BrainInsight3D: 3D fMRI
  Visualization \& Segmentation}

\author{Nada Elmasry}

\date{\today}

\maketitle

\pagenumbering{roman}

\section{Revision History}

\begin{tabularx}{\textwidth}{p{3cm}p{2cm}X}
  \toprule {\bf Date} & {\bf Version} & {\bf Notes} \\
  \midrule
  Date 1              & 1.0           & Notes       \\
  Date 2              & 1.1           & Notes       \\
  \bottomrule
\end{tabularx}

~\newpage

\section{Symbols, Abbreviations and Acronyms}

See SRS Documentation at \url{https://github.com/NadaElmasry/3D-Image-Reconstruction-Segmentation/blob/main/docs/SRS/SRS.pdf}


\newpage

\tableofcontents

\newpage

\pagenumbering{arabic}

\section{Introduction}

The following document details the Module Interface Specifications for BrainInsight3D: 3D fMRI
Visualization \& Segmentation.


Complementary documents include the System Requirement Specifications
and Module Guide.  The full documentation and implementation can be
found at \url{https://github.com/NadaElmasry/3D-Image-Reconstruction-Segmentation}.

\section{Notation}


The structure of the MIS for modules comes from \citet{HoffmanAndStrooper1995},
with the addition that template modules have been adapted from
\cite{GhezziEtAl2003}.  The mathematical notation comes from Chapter 3 of
\citet{HoffmanAndStrooper1995}.  For instance, the symbol := is used for a
multiple assignment statement and conditional rules follow the form $(c_1
  \Rightarrow r_1 | c_2 \Rightarrow r_2 | ... | c_n \Rightarrow r_n )$.

The following table summarizes the primitive data types used by \progname.

\begin{center}
  \renewcommand{\arraystretch}{1.2}
  \noindent
  \begin{tabular}{l l p{7.5cm}}
    \toprule
    \textbf{Data Type} & \textbf{Notation} & \textbf{Description}                                             \\
    \midrule
    character          & char              & a single symbol or digit                                         \\
    integer            & $\mathbb{Z}$      & a number without a fractional component in (-$\infty$, $\infty$) \\
    natural number     & $\mathbb{N}$      & a number without a fractional component in [1, $\infty$)         \\
    real               & $\mathbb{R}$      & any number in (-$\infty$, $\infty$)                              \\
    \bottomrule
  \end{tabular}
\end{center}

\noindent
The specification of \progname \ uses some derived data types: sequences, strings, and
tuples. Sequences are lists filled with elements of the same data type. Strings
are sequences of characters. Tuples contain a list of values, potentially of
different types. In addition, \progname \ uses functions, which
are defined by the data types of their inputs and outputs. Local functions are
described by giving their type signature followed by their specification.

\section{Module Decomposition}

The following table is taken directly from the Module Guide document for this project.

\begin{table}[h!]
  \centering
  \begin{tabular}{p{0.3\textwidth} p{0.6\textwidth}}
    \toprule
    \textbf{Level 1}                                      & \textbf{Level 2}                         \\
    \midrule

    {Hardware-Hiding Module}                              & ~                                        \\
    \midrule

    \multirow{7}{0.3\textwidth}{Behaviour-Hiding Module}  & Input Format Module                      \\
                                                          & Input Verification Module                \\
                                                          & Data Preprocessing Module                \\
                                                          & 2D Slice Module                          \\
                                                          & 3D Volume Module                         \\
                                                          & Segmentation Module                      \\
                                                          & User Interface Module                    \\
                                                          & Main Program Module                      \\

    \midrule

    \multirow{3}{0.3\textwidth}{Software Decision Module} & 2D Data Visualization Module             \\
                                                          & 3D Data Visualization Module             \\
                                                          & User Interface Rendering Module          \\
                                                          & Segmentation Algorithm Processing Module \\
                                                          & Optimization Module                      \\
    \bottomrule
  \end{tabular}
  \caption{Module Hierarchy}
  \label{TblMH}
\end{table}

\newpage
~\newpage

\section{MIS of Input Format Module} \label{IF}


\subsection{Module}

InputFormat

\subsection{Uses}
None
\subsection{Syntax}

\subsubsection{Exported Constants}
None

\subsubsection{Exported Access Programs}

\begin{enumerate}
  \item read\_scan
        \begin{itemize}
          \item \textbf{Input:} NIfTI file containing scan meta-data and scan of dimensions $ 1 \times 3 \times 3 \times 3$ containg the input scan voxel values at
                each timepoint.
          \item \textbf{Output:} scan meta-data in a tuple format and scan voxel values in NumPy ndarray of dimensions $ 1 \times 3 \times 3 \times 3$.
          \item \textbf{Exception:} "InvalidFormatError" if the input file is not in NIfTI format. "FileNotFoundError" if the input file cannot be found in file path.
                "CorruptFileError" if the input scan is corrupted.
        \end{itemize}
\end{enumerate}

\subsection{Semantics}

\subsubsection{State Variables}

CurrentScanData: Carries the input scan-metadata and matrix with voxel values.

\subsubsection{Environment Variables}
None

\subsubsection{Assumptions}
None
\subsubsection{Access Routine Semantics}

\noindent read\_scan(scan\_file):
\begin{itemize}
  \item transition: accepts NIfTI files as input and changes the state variable
        CurrentScanData to the most recent successfully read NIfTI scan file.
  \item output: scan meta-data (tuple), scan voxel values (NumPy ndarray of dimensions $1x3x3x3$)
  \item exception: "InvalidFormatError" if the input file is not in NIfTI format. "FileNotFoundError" if the input file cannot be found in file path.
        "CorruptFileError" if the input scan is corrupted.
\end{itemize}



\subsubsection{Local Functions}

None

\newpage

\section{MIS of Input Verification Module} \label{IV}


\subsection{Module}

Input Verification

\subsection{Uses}
ImageFormatModule

\subsection{Syntax}

\subsubsection{Exported Constants}
None

\subsubsection{Exported Access Programs}

\begin{enumerate}
  \item validate\_scan
        \begin{itemize}
          \item \textbf{Input:} scan meta-data in a tuple format and scan voxel values in NumPy ndarray of dimensions $ 1 \times 3 \times 3 \times 3$.
          \item \textbf{Output:} None.
          \item \textbf{Exception:} "MissingMetaDataError" if metadata is missing attributes required by the software
                for correct rendering. "MismatchedDimensionsError" if scan dimensions are not as expected by the software application.
                "InvalidDataRange" if input scan data isn't in the normal fMRI scan ranges or if all values is zero.
        \end{itemize}
\end{enumerate}

\subsection{Semantics}

\subsubsection{State Variables}

None

\subsubsection{Environment Variables}

None

\subsubsection{Assumptions}
Scan data has successfully passed the checks in ImageFormatModule.

\subsubsection{Access Routine Semantics}

\noindent validate\_scan(meta\_data, scan\_array):
\begin{itemize}
  \item transition: -
  \item output: None.
  \item exception: "MissingMetaDataError" if metadata is missing attributes required by the software
        for correct rendering. "MismatchedDimensionsError" if scan dimensions are not as expected by the software application.
        "InvalidDataRange" if input scan data isn't in the normal fMRI scan ranges or if all values is zero.

\end{itemize}



\subsubsection{Local Functions}

None.

\newpage

\section{MIS of Data Preprocessing Module} \label{DP}


\subsection{Module}

DataProc

\subsection{Uses}
None

\subsection{Syntax}

\subsubsection{Exported Constants}
None
\subsubsection{Exported Access Programs}

\begin{enumerate}
  \item align\_scan
        \begin{itemize}
          \item \textbf{Input:} NumPy ndarray of scan voxel data.
          \item \textbf{Output:} NumPy ndarray of scan data aligned.
          \item \textbf{Exception:} "ImageAlignmentError" if there is an error that causes the
                alignment algorithm to fail.
        \end{itemize}
  \item normalize\_scan
        \begin{itemize}
          \item \textbf{Input:} NumPy ndarray of scan voxel data.
          \item \textbf{Output:} NumPy ndarray of scan voxel data normalized with zero mean and unit standard deviation.
          \item \textbf{Exception:} "ImageNormalizationError" if there is an error that causes the
                normalization algorithm to fail.
        \end{itemize}
  \item remove\_noise
        \begin{itemize}
          \item \textbf{Input:} NumPy ndarray of scan voxel data.
          \item \textbf{Output:} NumPy ndarray of scan voxel data with reduced noise.
          \item \textbf{Exception:} "ImageNoiseRemovalError" if there is an error that causes the
                noise removal algorithm to fail.
        \end{itemize}
  \item register\_scan
        \begin{itemize}
          \item \textbf{Input:} NumPy ndarray of scan voxel data.
          \item \textbf{Output:} NumPy ndarray of registered scan voxel data.
          \item \textbf{Exception:} "ImageRegistrationError" if there is an error that causes the
                registration algorithm to fail.
        \end{itemize}
  \item enhance\_contrast
        \begin{itemize}
          \item \textbf{Input:} NumPy ndarray of scan voxel data.
          \item \textbf{Output:} NumPy ndarray of scan voxel data with enhanced contrast.
          \item \textbf{Exception:} "ImageContrastEnhancementError" if there is an error that causes the
                contrast enhancement algorithm to fail.
        \end{itemize}
\end{enumerate}

\subsection{Semantics}

\subsubsection{State Variables}
\begin{itemize}
  \item isAligned: boolean to indicate whether the scan has passed
        through the image alignment step of the preproceesing pipeline.
  \item isRegistered: boolean to indicate whether the scan has passed
        through the image registeration step of the preproceesing pipeline.
  \item isNormalized: boolean to indicate whether the scan has passed
        through the image normalization step of the preproceesing pipeline.
  \item isDenoised: boolean to indicate whether the scan has passed
        through the image denoising step of the preproceesing pipeline.
  \item isContrastEnhanced: boolean to indicate whether the scan has passed
        through the contrast enhancement step of the preproceesing pipeline.
\end{itemize}

\subsubsection{Environment Variables}

None.

\subsubsection{Assumptions}

Scan data has successfully passed the tests in the InputVerificationModule.
\subsubsection{Access Routine Semantics}

\noindent align\_scan(scan\_array):
\begin{itemize}
  \item transition: performs image alignment process on all slices of
        the input scan volumes and sets the isAligned state variable to true
        if the function terminates successfully.
  \item output: NumPy ndarray of scan data aligned.
  \item exception: "ImageAlignmentError" if there is an error that causes the
        alignment algorithm to fail.
\end{itemize}
\noindent register\_scan(scan\_array):
\begin{itemize}
  \item transition: performs image registration process on all slices of
        the input scan volumes and sets the isRegistered state variable to true
        if the function terminates successfully.
  \item output: NumPy ndarray of registered scan data.
  \item exception: "ImageRegistrationError" if there is an error that causes the
        registration algorithm to fail.
\end{itemize}
\noindent normalized\_scan(scan\_array):
\begin{itemize}
  \item transition: performs image normalization process on all slices of
        the input scan volumes and sets the isNormalized state variable to true
        if the function terminates successfully.
  \item output: NumPy ndarray of normalized scan data.
  \item exception: "ImageNormalizationError" if there is an error that causes the
        normalization algorithm to fail.
\end{itemize}
\noindent remove\_noise(scan\_array):
\begin{itemize}
  \item transition: performs noise removal process on all slices of
        the input scan volumes and sets the isDenoised state variable to true
        if the function terminates successfully.
  \item output:NumPy ndarray of scan voxel data with reduced noise.
  \item exception: "ImageNoiseRemovalError" if there is an error that causes the
        noise removal algorithm to fail.
\end{itemize}
\noindent enhance\_contrast(scan\_array):
\begin{itemize}
  \item transition: performs contrast enhancement process on all slices of
        the input scan volumes and sets the isContrastEnhanced state variable to true
        if the function terminates successfully.
  \item output: NumPy ndarray of scan data with enhanced contrast.
  \item exception: "ImageContrastEnhancementError" if there is an error that causes the
        contrast enhancement algorithm to fail.
\end{itemize}

\subsubsection{Local Functions}

\newpage


\section{MIS of 2D Slice Module} \label{2DS}


\subsection{Module}

2DSlicer
\subsection{Uses}
DataPreprocessingModule

\subsection{Syntax}

\subsubsection{Exported Constants}

\subsubsection{Exported Access Programs}

\begin{enumerate}
  \item get\_slice\_data
        \begin{itemize}
          \item \textbf{Input:} NumPy ndarray of scan voxel data.
          \item \textbf{Output:} NumPy ndarray of slice data where each slice is a two-dimensional NumPy ndarray.
          \item \textbf{Exception:} "SlicingError" if the slicing algorithm cannot perform slicing properly due
                to input incorrect dimensions or an error or condition causes the slicing algorithm to terminate without completing the slicing process.
        \end{itemize}

\end{enumerate}


\subsection{Semantics}

\subsubsection{State Variables}

None

\subsubsection{Environment Variables}

None

\subsubsection{Assumptions}

Input scan is preprocessed and is in correct format.

\subsubsection{Access Routine Semantics}

\noindent get\_slice\_data(scan\_array):
\begin{itemize}
  \item transition: -
  \item output: NumPy ndarray of slice data where each slice is a two-dimensional NumPy ndarray.
  \item exception: "SlicingError" if the slicing algorithm cannot perform slicing properly due
        to input incorrect dimensions or an error or condition causes the slicing algorithm to terminate without completing the slicing process.
\end{itemize}


\subsubsection{Local Functions}
None.

\newpage
\section{MIS of 3D Volume Module} \label{3DV}


\subsection{Module}

3DVolumeReconstruct

\subsection{Uses}
Data Preprocessing Module

\subsection{Syntax}

\subsubsection{Exported Constants}

\subsubsection{Exported Access Programs}

\begin{enumerate}
  \item get\_volume\_data
        \begin{itemize}
          \item \textbf{Input:} NumPy ndarray of scan voxel data.
          \item \textbf{Output:} NumPy ndarray of scan voxel data after compression and mapping.
          \item \textbf{Exception:} "MappingError" if the voxel mapping algorithm fails due to scan properties.
                "CompressionError" if the input scan cannot be compressed to the desired size without significant loss of data.
                "ReconstructionError" if the 3D volume reconstruction fails to terminate successfully and return non-corrupt volume data.
        \end{itemize}
\end{enumerate}


\subsection{Semantics}

\subsubsection{State Variables}

None

\subsubsection{Environment Variables}

None
\subsubsection{Assumptions}

Input scan is preprocessed and is in correct format.

\subsubsection{Access Routine Semantics}

\noindent get\_volume\_data(scan\_array):
\begin{itemize}
  \item transition: -
  \item output: NumPy ndarray of scan voxel data after compression and mapping.
  \item exception: "MappingError" if the voxel mapping algorithm fails due to scan properties.
        "CompressionError" if the input scan cannot be compressed to the desired size without significant loss of data.
        "ReconstructionError" if the 3D volume reconstruction fails to terminate successfully and return non-corrupt volume data.
\end{itemize}


\subsubsection{Local Functions}

None.


\newpage
\section{MIS of Segmentation Module} \label{Seg}

\subsection{Module}

Seg

\subsection{Uses}
Data Preprocessing Module

\subsection{Syntax}

\subsubsection{Exported Constants}

\subsubsection{Exported Access Programs}

\begin{enumerate}
  \item segment\_volume
        \begin{itemize}
          \item \textbf{Input:} NumPy ndarray of scan voxel data.
          \item \textbf{Output:} NumPy ndarray of scan voxel data and the associated segmentation mask value for each voxel.
          \item \textbf{Exception:} "SegmentationError" if model fails to produce segmentation mask from input scan due to
                model internal error or network connectivity error if the connection with the web application has been reset.
        \end{itemize}
\end{enumerate}

\subsection{Semantics}

\subsubsection{State Variables}

\begin{itemize}
  \item InputScan:
  \item SegmentationMask:
\end{itemize}

\subsubsection{Environment Variables}

None.

\subsubsection{Assumptions}

Input data has passed through the data preprocessing pipeline.
\subsubsection{Access Routine Semantics}

\noindent get\_volume\_data(scan\_array):
\begin{itemize}
  \item transition: -
  \item output:  NumPy ndarray of scan voxel data and the associated segmentation mask value for each voxel.
  \item exception: "SegmentationError" if model fails to produce segmentation mask from input scan due to
        model internal error or network connectivity error if the connection with the web application has been reset.
\end{itemize}


\subsubsection{Local Functions}

None.

\newpage

\section{MIS of User Interface Module} \label{UI} \wss{Use labels for
  cross-referencing}


\subsection{Module}

UI

\subsection{Uses}

Main Program Module


\subsection{Syntax}

\subsubsection{Exported Constants}

\subsubsection{Exported Access Programs}



\subsection{Semantics}

\subsubsection{State Variables}

DisplayedTab: Indicates which tab (3D volume visualization, 2D slices visualization) is currently
displayed.



\subsubsection{Environment Variables}

None.

\subsubsection{Assumptions}

None.

\subsubsection{Access Routine Semantics}




\subsubsection{Local Functions}

None.


\newpage
% \section{MIS of Main Program Module} \label{MP}

% \wss{You can reference SRS labels, such as R\ref{R_Inputs}.}

% \wss{It is also possible to use \LaTeX for hypperlinks to external documents.}

% \subsection{Module}

% \wss{Short name for the module}

% \subsection{Uses}


% \subsection{Syntax}

% \subsubsection{Exported Constants}

% \subsubsection{Exported Access Programs}

% \begin{center}
%   \begin{tabular}{p{2cm} p{4cm} p{4cm} p{2cm}}
%     \hline
%     \textbf{Name}    & \textbf{In} & \textbf{Out} & \textbf{Exceptions} \\
%     \hline
%     \wss{accessProg} & -           & -            & -                   \\
%     \hline
%   \end{tabular}
% \end{center}

% \subsection{Semantics}

% \subsubsection{State Variables}

% \wss{Not all modules will have state variables.  State variables give the module
%   a memory.}

% \subsubsection{Environment Variables}

% \wss{This section is not necessary for all modules.  Its purpose is to capture
%   when the module has external interaction with the environment, such as for a
%   device driver, screen interface, keyboard, file, etc.}

% \subsubsection{Assumptions}

% \wss{Try to minimize assumptions and anticipate programmer errors via
%   exceptions, but for practical purposes assumptions are sometimes appropriate.}

% \subsubsection{Access Routine Semantics}

% \noindent \wss{accessProg}():
% \begin{itemize}
%   \item transition: \wss{if appropriate}
%   \item output: \wss{if appropriate}
%   \item exception: \wss{if appropriate}
% \end{itemize}

% \wss{A module without environment variables or state variables is unlikely to
%   have a state transition.  In this case a state transition can only occur if
%   the module is changing the state of another module.}

% \wss{Modules rarely have both a transition and an output.  In most cases you
%   will have one or the other.}

% \subsubsection{Local Functions}

% \wss{As appropriate} \wss{These functions are for the purpose of specification.
%   They are not necessarily something that is going to be implemented
%   explicitly.  Even if they are implemented, they are not exported; they only
%   have local scope.}

% \newpage
% \section{MIS of Data Visualization} \label{DV}

% \wss{You can reference SRS labels, such as R\ref{R_Inputs}.}

% \wss{It is also possible to use \LaTeX for hypperlinks to external documents.}

% \subsection{Module}

% \wss{Short name for the module}

% \subsection{Uses}


% \subsection{Syntax}

% \subsubsection{Exported Constants}

% \subsubsection{Exported Access Programs}

% \begin{center}
%   \begin{tabular}{p{2cm} p{4cm} p{4cm} p{2cm}}
%     \hline
%     \textbf{Name}    & \textbf{In} & \textbf{Out} & \textbf{Exceptions} \\
%     \hline
%     \wss{accessProg} & -           & -            & -                   \\
%     \hline
%   \end{tabular}
% \end{center}

% \subsection{Semantics}

% \subsubsection{State Variables}

% \wss{Not all modules will have state variables.  State variables give the module
%   a memory.}

% \subsubsection{Environment Variables}

% \wss{This section is not necessary for all modules.  Its purpose is to capture
%   when the module has external interaction with the environment, such as for a
%   device driver, screen interface, keyboard, file, etc.}

% \subsubsection{Assumptions}

% \wss{Try to minimize assumptions and anticipate programmer errors via
%   exceptions, but for practical purposes assumptions are sometimes appropriate.}

% \subsubsection{Access Routine Semantics}

% \noindent \wss{accessProg}():
% \begin{itemize}
%   \item transition: \wss{if appropriate}
%   \item output: \wss{if appropriate}
%   \item exception: \wss{if appropriate}
% \end{itemize}

% \wss{A module without environment variables or state variables is unlikely to
%   have a state transition.  In this case a state transition can only occur if
%   the module is changing the state of another module.}

% \wss{Modules rarely have both a transition and an output.  In most cases you
%   will have one or the other.}

% \subsubsection{Local Functions}

% \wss{As appropriate} \wss{These functions are for the purpose of specification.
%   They are not necessarily something that is going to be implemented
%   explicitly.  Even if they are implemented, they are not exported; they only
%   have local scope.}

% \newpage
% \section{MIS of User Interface Rendering Module} \label{UIR}

% \wss{You can reference SRS labels, such as R\ref{R_Inputs}.}

% \wss{It is also possible to use \LaTeX for hypperlinks to external documents.}

% \subsection{Module}

% \wss{Short name for the module}

% \subsection{Uses}


% \subsection{Syntax}

% \subsubsection{Exported Constants}

% \subsubsection{Exported Access Programs}

% \begin{center}
%   \begin{tabular}{p{2cm} p{4cm} p{4cm} p{2cm}}
%     \hline
%     \textbf{Name}    & \textbf{In} & \textbf{Out} & \textbf{Exceptions} \\
%     \hline
%     \wss{accessProg} & -           & -            & -                   \\
%     \hline
%   \end{tabular}
% \end{center}

% \subsection{Semantics}

% \subsubsection{State Variables}

% \wss{Not all modules will have state variables.  State variables give the module
%   a memory.}

% \subsubsection{Environment Variables}

% \wss{This section is not necessary for all modules.  Its purpose is to capture
%   when the module has external interaction with the environment, such as for a
%   device driver, screen interface, keyboard, file, etc.}

% \subsubsection{Assumptions}

% \wss{Try to minimize assumptions and anticipate programmer errors via
%   exceptions, but for practical purposes assumptions are sometimes appropriate.}

% \subsubsection{Access Routine Semantics}

% \noindent \wss{accessProg}():
% \begin{itemize}
%   \item transition: \wss{if appropriate}
%   \item output: \wss{if appropriate}
%   \item exception: \wss{if appropriate}
% \end{itemize}

% \wss{A module without environment variables or state variables is unlikely to
%   have a state transition.  In this case a state transition can only occur if
%   the module is changing the state of another module.}

% \wss{Modules rarely have both a transition and an output.  In most cases you
%   will have one or the other.}

% \subsubsection{Local Functions}

% \wss{As appropriate} \wss{These functions are for the purpose of specification.
%   They are not necessarily something that is going to be implemented
%   explicitly.  Even if they are implemented, they are not exported; they only
%   have local scope.}

% \newpage
% \section{MIS of Segmentation Algorithm Processing Module} \label{SegR}

% \wss{You can reference SRS labels, such as R\ref{R_Inputs}.}

% \wss{It is also possible to use \LaTeX for hypperlinks to external documents.}

% \subsection{Module}

% \wss{Short name for the module}

% \subsection{Uses}


% \subsection{Syntax}

% \subsubsection{Exported Constants}

% \subsubsection{Exported Access Programs}

% \begin{center}
%   \begin{tabular}{p{2cm} p{4cm} p{4cm} p{2cm}}
%     \hline
%     \textbf{Name}    & \textbf{In} & \textbf{Out} & \textbf{Exceptions} \\
%     \hline
%     \wss{accessProg} & -           & -            & -                   \\
%     \hline
%   \end{tabular}
% \end{center}

% \subsection{Semantics}

% \subsubsection{State Variables}

% \wss{Not all modules will have state variables.  State variables give the module
%   a memory.}

% \subsubsection{Environment Variables}

% \wss{This section is not necessary for all modules.  Its purpose is to capture
%   when the module has external interaction with the environment, such as for a
%   device driver, screen interface, keyboard, file, etc.}

% \subsubsection{Assumptions}

% \wss{Try to minimize assumptions and anticipate programmer errors via
%   exceptions, but for practical purposes assumptions are sometimes appropriate.}

% \subsubsection{Access Routine Semantics}

% \noindent \wss{accessProg}():
% \begin{itemize}
%   \item transition: \wss{if appropriate}
%   \item output: \wss{if appropriate}
%   \item exception: \wss{if appropriate}
% \end{itemize}

% \wss{A module without environment variables or state variables is unlikely to
%   have a state transition.  In this case a state transition can only occur if
%   the module is changing the state of another module.}

% \wss{Modules rarely have both a transition and an output.  In most cases you
%   will have one or the other.}

% \subsubsection{Local Functions}

% \wss{As appropriate} \wss{These functions are for the purpose of specification.
%   They are not necessarily something that is going to be implemented
%   explicitly.  Even if they are implemented, they are not exported; they only
%   have local scope.}




% \newpage
% \section{MIS of Optimization Module} \label{Op}

% \wss{You can reference SRS labels, such as R\ref{R_Inputs}.}

% \wss{It is also possible to use \LaTeX for hypperlinks to external documents.}

% \subsection{Module}

% \wss{Short name for the module}

% \subsection{Uses}


% \subsection{Syntax}

% \subsubsection{Exported Constants}

% \subsubsection{Exported Access Programs}

% \begin{center}
%   \begin{tabular}{p{2cm} p{4cm} p{4cm} p{2cm}}
%     \hline
%     \textbf{Name}    & \textbf{In} & \textbf{Out} & \textbf{Exceptions} \\
%     \hline
%     \wss{accessProg} & -           & -            & -                   \\
%     \hline
%   \end{tabular}
% \end{center}

% \subsection{Semantics}

% \subsubsection{State Variables}

% \wss{Not all modules will have state variables.  State variables give the module
%   a memory.}

% \subsubsection{Environment Variables}

% \wss{This section is not necessary for all modules.  Its purpose is to capture
%   when the module has external interaction with the environment, such as for a
%   device driver, screen interface, keyboard, file, etc.}

% \subsubsection{Assumptions}

% \wss{Try to minimize assumptions and anticipate programmer errors via
%   exceptions, but for practical purposes assumptions are sometimes appropriate.}

% \subsubsection{Access Routine Semantics}

% \noindent \wss{accessProg}():
% \begin{itemize}
%   \item transition: \wss{if appropriate}
%   \item output: \wss{if appropriate}
%   \item exception: \wss{if appropriate}
% \end{itemize}

% \wss{A module without environment variables or state variables is unlikely to
%   have a state transition.  In this case a state transition can only occur if
%   the module is changing the state of another module.}

% \wss{Modules rarely have both a transition and an output.  In most cases you
%   will have one or the other.}

% \subsubsection{Local Functions}

% \wss{As appropriate} \wss{These functions are for the purpose of specification.
%   They are not necessarily something that is going to be implemented
%   explicitly.  Even if they are implemented, they are not exported; they only
%   have local scope.}





\newpage

\bibliographystyle {plainnat}
\bibliography {../../../refs/ReferencesMIS}


\end{document}