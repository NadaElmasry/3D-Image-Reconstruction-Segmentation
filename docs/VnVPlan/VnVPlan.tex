\documentclass[12pt, titlepage]{article}

\usepackage{booktabs}
\usepackage{tabularx}
\usepackage{hyperref}
\hypersetup{
    colorlinks,
    citecolor=blue,
    filecolor=black,
    linkcolor=red,
    urlcolor=blue
}
\usepackage[round]{natbib}

%% Comments

\usepackage{color}

\newif\ifcomments\commentstrue %displays comments
%\newif\ifcomments\commentsfalse %so that comments do not display

\ifcomments
\newcommand{\authornote}[3]{\textcolor{#1}{[#3 ---#2]}}
\newcommand{\todo}[1]{\textcolor{red}{[TODO: #1]}}
\else
\newcommand{\authornote}[3]{}
\newcommand{\todo}[1]{}
\fi

\newcommand{\wss}[1]{\authornote{blue}{SS}{#1}} 
\newcommand{\plt}[1]{\authornote{magenta}{TPLT}{#1}} %For explanation of the template
\newcommand{\an}[1]{\authornote{cyan}{Author}{#1}}

%% Common Parts

\newcommand{\progname}{ProgName} % PUT YOUR PROGRAM NAME HERE
\newcommand{\authname}{Team \#, Team Name
\\ Student 1 name
\\ Student 2 name
\\ Student 3 name
\\ Student 4 name} % AUTHOR NAMES                  

\usepackage{hyperref}
    \hypersetup{colorlinks=true, linkcolor=blue, citecolor=blue, filecolor=blue,
                urlcolor=blue, unicode=false}
    \urlstyle{same}
                                


\begin{document}

\title{System Verification and Validation Plan for BrainInsight3D: 3D fMRI
  Visualization \& Segmentation}
\author{Nada Elmasry}
\date{\today}

\maketitle

\pagenumbering{roman}

\section*{Revision History}

\begin{tabularx}{\textwidth}{p{3cm}p{2cm}X}
  \toprule {\bf Date} & {\bf Version} & {\bf Notes} \\
  \midrule
  Date 1              & 1.0           & Notes       \\
  Date 2              & 1.1           & Notes       \\
  \bottomrule
\end{tabularx}

~\\

\newpage

\tableofcontents

% \listoftables
% \wss{Remove this section if it isn't needed}

% \listoffigures
% \wss{Remove this section if it isn't needed}

\newpage

\section{Symbols, Abbreviations, and Acronyms}

\renewcommand{\arraystretch}{1.2}
\begin{tabular}{l l}
  \toprule
  \textbf{symbol} & \textbf{description} \\
  \midrule
  T               & Test                 \\
  \bottomrule
\end{tabular}\\

See complete list of symbols and abbreviations at \cite{BrainInsightSRS2024}
% \wss{symbols, abbreviations, or acronyms --- you can simply reference the SRS
%   \citep{SRS} tables, if appropriate}

% \wss{Remove this section if it isn't needed}

\newpage

\pagenumbering{arabic}

\section{Introduction}
This document outlines the System Verification and Validation Plan (VnV) for BrainInsight3D: 3D fMRI
Visualization \& Segmentation project. The VnV plan outlines the verification and validation tests
that need to be conducted in order for the software product to be have satisfied its requirements.
Verification is done by testing individual components and multiple sub-components of the software products together
to verify that they work correctly. Validation is done to verify that the finished software product meets the user
needs and requirements.

the document is divided into five main parts, section\ref{section3} presents the general information
and objectives of the project, section\ref{section4} discusses methods of testing to achieve project's
objective, section\ref{section5} discusses the system test descriptions including functional and nonfuctional
requirements tests, and section\ref{section6} discusses the unit testing of the systems components.



\section{General Information} \label{section3}

\subsection{Summary}

This document provides the software specification requirements(SRS) for the
BrainInsight 3D project. The project offers 3D and 2D visualization and segmentation
of fMRI brain scans with time and functional activity tracking in 2D through a
web application that enables high-resolution visualization and segmentation.



% \wss{Say what software is being tested.  Give its name and a brief overview of
%   its general functions.}

\subsection{Objectives}

the objective of this document is to build confidience in the software correctness and
usability by producing results comparable to current solutions with an easy to use interface
for better interaction flow and usability. A secondary goal of this project is to make sure the project
is extensible, this can be achieved by ensuring code quality and maintanibility to enable extensibility by multiple developers.
Parts of the code will rely on external libraries for operations like image input and preprocessing, and bulding and hosting the
user interface as a web application. While it would be beneficial to verify the correctness
of these libraries, this is out of the scope of the project and it will be assumed
these external libraries will have been verified by their implementation teams.
% \wss{State what is intended to be accomplished.  The objective will be around
%   the qualities that are most important for your project.  You might have
%   something like: ``build confidence in the software correctness,''
%   ``demonstrate adequate usability.'' etc.  You won't list all of the qualities,
%   just those that are most important.}

% \wss{You should also list the objectives that are out of scope.  You don't have
%   the resources to do everything, so what will you be leaving out.  For instance,
%   if you are not going to verify the quality of usability, state this.  It is also
%   worthwhile to justify why the objectives are left out.}

% \wss{The objectives are important because they highlight that you are aware of
%   limitations in your resources for verification and validation.  You can't do everything,
%   so what are you going to prioritize?  As an example, if your system depends on an
%   external library, you can explicitly state that you will assume that external library
%   has already been verified by its implementation team.}

\subsection{Relevant Documentation}


There exists multiple documents which are crucial for in-depth understanding of the software being tested.
The documents are as follows:
\begin{itemize}
  \item Problem Statement \cite{BrainInsightProblemStatement2024}: states the problem the project is trying to solve and introduces the project's features.
  \item Software Requirements Specification (SRS) Document \cite{BrainInsightSRS2024}: States the assumptions, theoritical and implementation models,
        functional and nonfunctional requirements, and terminology that will be used throughout the project documentation
  \item Module Guide (MG) Document \cite{BrainInsightMG2024}: states the functionality and responsibilities of each module in the project.
  \item Module Interface Specification Document \cite{BrainInsightMIS2024}: States the functional design of each module in the project.
\end{itemize}

% \wss{Reference relevant documentation.  This will definitely include your SRS
%   and your other project documents (design documents, like MG, MIS, etc).  You
%   can include these even before they are written, since by the time the project
%   is done, they will be written.}


% \wss{Don't just list the other documents.  You should explain why they are relevant and
%   how they relate to your VnV efforts.}

\section{Plan} \label{section4}

% \wss{Introduce this section.   You can provide a roadmap of the sections to
%   come.}
This section includes the verification detailed plan for the documentation and
software of the Braininsight3D project. Section \ref{section4.1} introduces the verification and
validation team responsible for through implementation and validation of all aspects of the project's documentation
and software components, sections \ref{section4.2}, \ref{section4.3}, \ref{section4.4} discuss the verification
plan for the project's SRS, design, and VnV plans respectively. Sections \ref{section4.5}, \ref{section4.6}, and
\ref{section4.7} discuss the software implementation verification plan, automated testing and verification tools, and
software validation plan respectively. The primary aim of this section is to provide a complete description
of the plans which will be used to ensure the software abides to its requirements and achieves its objectives.
\subsection{Verification and Validation Team} \label{section4.1}

% \wss{Your teammates.  Maybe your supervisor.
%   You should do more than list names.  You should say what each person's role is
%   for the project's verification.  A table is a good way to summarize this information.}
Table \ref{VnVTeamTable} lists all VnV team members and their roles in the project.
~\newline

\noindent
\begin{minipage}{\textwidth}
  \renewcommand*{\arraystretch}{1.5}
  \begin{tabular}{| l | p{0.2\textwidth}|p{0.3\textwidth}|p{0.3\textwidth}|}
    \hline
    % \rowcolor[gray]{0.9} 
    \textbf{Name}     & \textbf{Document}      & \textbf{Role}              & \textbf{Description}                                                                 \\
    \hline
    Dr. Spencer Smith & SRS, VnV Plan, MG, MIS & Instructor/Expert Reviewer & Reviews all project's documentation, design, and functionality                       \\
    \hline
    Nada Elmasry      & SRS, Vnv Plan, MG, MIS & Author/Developer           & Author of all project's documents. Main developer and tester of the software product \\
    \hline
    Yi-Leng Cheng     & SRS, VnV Plan, MG, MIS & Domain Expert Reviewer     & Review all project's documents.                                                      \\
    \hline
    Waqar Awan        & SRS                    & SRS Secondary Reviewer     & Reviews the SRS document.                                                            \\
    \hline
    Kim Ying Wong     & VnV Plan               & Secondary Reviewer         & Reviews the VnV Plan document.                                                       \\
    \hline
    Morteza Mirzaei   & MG, MIS                & Secondary Reviewer         & Reviews the MG+MIS document.                                                         \\
    \hline
  \end{tabular}
  % \caption{Verification and Validation team}
  \label{VnVTeamTable}
\end{minipage}\\

\subsection{SRS Verification Plan} \label{section4.2}

% \wss{List any approaches you intend to use for SRS verification.  This may include
%   ad hoc feedback from reviewers, like your classmates, or you may plan for
%   something more rigorous/systematic.}

% \wss{Maybe create an SRS checklist?}

The SRS document will be reviewed by the instructor, the domain expert reviewer, and the secondary reviewer to ensure that the documents fulfills
the outlined requirements in the SRS checklist \cite{SRSChecklist}. The reviewers will provide their feedback through GitHub issues made by the author for SRS review.
It is the responsibility of the author to incorporate the reviewers feedback and suggestions, and to address any possible issues in the SRS documentation.

\subsection{Design Verification Plan} \label{section4.3}
The software design will be designed with a focus on usability, accuracy, and performance. The design documentation will include the Module Guide (MG) plan and the Module Interface Specification (MIS) plan.
The MG and MIS plans will be reviewed by the instructor, the domain expert reviewer, and the secondary reviewer to ensure that the documents fulfills
the outlined requirements in the MG checklist \cite{BrainInsightMG2024}and MIS checklist \cite{BrainInsightMIS2024}. The reviewers will provide their feedback through GitHub issues made by the author for MG and MIS review.
It is the responsibility of the author to incorporate the reviewers feedback and suggestions, and to address any possible issues in the design documentation.

% \wss{Plans for design verification}

% \wss{The review will include reviews by your classmates}

% \wss{Create a checklists?}

\subsection{Verification and Validation Plan Verification Plan} \label{section4.4}

The VnV plan will be continously updated throughout the project implementation phase.
The VnV team will check whether the documented testing and verification process have been accomplished and the software functional and nonfunctional
requirements have been fulfilled. The reviewers will provide their feedback through GitHub issues made by the author for the VnV plan review.
It is the responsibility of the author to incorporate the reviewers feedback and suggestions, and to address any possible issues in the VnV documentation.


% \wss{The verification and validation plan is an artifact that should also be
%   verified.  Techniques for this include review and mutation testing.}

% \wss{The review will include reviews by your classmates}

% \wss{Create a checklists?}

% \subsection{Implementation Verification Plan} \label{section4.5}

% \wss{You should at least point to the tests listed in this document and the unit
%   testing plan.}

% \wss{In this section you would also give any details of any plans for static
%   verification of the implementation.  Potential techniques include code
%   walkthroughs, code inspection, static analyzers, etc.}
\subsection{Implementation Verification Plan} \label{section4.5}

In this section, we outline the plan for verifying the implementation of our software system.
The verification process will ensure that the software meets the specified requirements and functions as intended.

\subsubsection{Dynamic Testing}
We will conduct a series of dynamic tests which include unit testing and integration testing. Due to the time constraints of the project, system testing and acceptance testing will not be conducted.
The unit testing plan is detailed in Section \ref{section6} of the Test Plan document, where each module of the software will be tested individually to ensure correct behavior.
Integration testing will ensure that multiple modules are well integrated and can communicate and operate together correctly.

\subsubsection{Static Verification}
In addition to dynamic testing, we will employ various static verification techniques to examine the source code. These techniques include:

\begin{itemize}
  \item \textbf{Code Walkthroughs:} The development team will conduct regular code walkthroughs, where the source code is reviewed to identify potential issues and ensure adherence to coding standards.

  \item \textbf{Code Inspections:} Formal code inspections will be carried out to systematically examine the code for defects, compliance with design and coding guidelines, and other quality attributes.

  \item \textbf{Static Analyzers:} We will use static analysis tools such as ESLint to automatically analyze the code for common programming errors, code smells, and to enforce coding standards.

\end{itemize}

These static verification techniques will complement our dynamic testing efforts and help ensure the reliability and quality of the software implementation.

\subsection{Automated Testing and Verification Tools} \label{section4.6}

The project will be maintained using CI/CD pipeline implemented using Docker and GitHub actions. GitHub Actions offers flexibility and customization when building CI/CD pipelines which is
cruical for our project as it have many overlapping components. The SRS document contains an introduction to the different components of the project \cite{BrainInsightSRS2024}. GitHub actions can
run tests by comparing expected outcomes with current outcomes from the software and directly build and deploy the software after each successful update. The individual tests will be written in Pytest library.
\subsection{Software Validation Plan} \label{section4.7}

Software Validation Plan is out of scope for BrainInsight3D due to the time constraints and limited resources of the project.

\section{System Test Description} \label{section5}

\subsection{Tests for Functional Requirements}

This sections presents the planned tests for the functional requirements for BrainInsight3D stated in SRS Document
\cite{BrainInsightSRS2024} in Section 5.1. There are six functional requirements for BrainInsight3D,
R1 is related to input, R2 to R4 are related to image processing and visualization, and R5 and R6 are
related to segmentation. Section \ref{section5.1} presents the tests for input verification, section \ref{section5.2} presents
the tests for image processing and visualization, and section \ref{section5.3} presents the tests for image segmentation.

% \wss{Subsets of the tests may be in related, so this section is divided into
%   different areas.  If there are no identifiable subsets for the tests, this
%   level of document structure can be removed.}

% \wss{Include a blurb here to explain why the subsections below
%   cover the requirements.  References to the SRS would be good here.}

\subsubsection{Input Testing} \label{section5.1}

% \wss{It would be nice to have a blurb here to explain why the subsections below
%   cover the requirements.  References to the SRS would be good here.  If a section
%   covers tests for input constraints, you should reference the data constraints
%   table in the SRS.}

The purpose of the tests in this section is to ensure the input scan is non-corrupt and in
the correct format in order to be viable to be used in the subsequent processing steps.

\paragraph{Input Scan Testing}

\begin{enumerate}

  \item{test-input-scan-format}

        Control: Automatic

        Initial State: System waiting to receive input scan

        Input: Input Scan in NIfTI format

        Output: A message signaling that the scan was read successfully or a detailed error message
        with the complete error log.

        Test Case Derivation: This test is conducted when the user uploads a new scan into the software application.
        The test ensures that the scan is in the correct format and non-corrupt before passing it to the preprocessing and
        visualization steps.

        How test will be performed: The input scan will be read by a library function, the function will return an output
        based on whether the scan was read successfully or not. The return message will then guide the system to print the success message
        or the failure message with the detailed error log.

\end{enumerate}

\subsubsection{Preprocessing and Visualization} \label{section5.2}

\begin{enumerate}

  \item{test-image-processing-slices-2d}

        Control: Automatic

        Initial State: Input Scan have been uploaded successfully in the supported format.

        Input: Input Scans with different orientations, noise levels, and contrast levels.

        Output: Processed scan slices with adjusted contrast and brightness levels, reduced noise, and the skull and neck
        details removed from the scan leaving only the brain area visible (region of interest).

        Test Case Derivation: This test validates a crucial step in the visualization pipeline, the image preprocessing step.
        Failure of this test will lead to failure of both the visualization and segmentation steps. The test validates that the
        input scan has successfully passed through all the preprocessing steps before the scan is visualized in 2D or 3D space, or passed to the
        segmentation model.

        How test will be performed: The input scan will be processed by the preprocessing function performing
        multiple operations as image registeration and smoothing to obtain the final result for visualization.
        If the function fails at any of the preprocessing steps, an error message will be printed highlighting
        the part where the error occurred along with the error log available from that processing step.

  \item{test-image-visualization-volume-3d}

        Control: Automatic

        Initial State: Input scans have successfully passed through the preprocessing pipeline.

        Input: Preprocessed input scans.

        Output: Reconstructed 3D volume of the input scan 2D slices.

        Test Case Derivation: This test validates that the 3D volume has been successfully reconstructed from the 2D slices
        for visualization in the web interface.

        How test will be performed: The preprocessed input scans will be used as input to the 3D reconstruction function that
        will return the 3D volume if the reconstruction was successful, otherwise it will return an error log explaining where exactly
        in the reconstruction step did the error occur.


\end{enumerate}
\subsubsection{Area of Testing3} \label{section5.3}

\begin{enumerate}

  \item{test-image-segmentation-model-load}

        Control: Automatic

        Initial State: System waiting for segmentation model to load

        Input: Segmentation Model file

        Output: A message signaling if the segmentation model has been loaded successfully or an error
        message with the exact error log explaining where and why the error occurred.

        Test Case Derivation: This test ensures that the segmentation model is successfully loaded
        and ready to receive input slices to perform the segmentation process.

        How test will be performed: The model file will be uploaded in .h5 format which will be read by
        the machine learning library(PyTorch) and return a message indicating the success or failure of loading the model file.

  \item{test-image-segmentation-mask-result}

        Control: Automatic

        Initial State: Segmentation model ready to receive input slice.

        Input: Processed slice image.

        Output: Mask image with the segmented brain areas.

        Test Case Derivation: This test case tests the performance of the segmentation model when
        receiving a preprocessed brain image slice for segmentation of the functional brain areas.
        The model should output a gray mask image with the segmented brain areas.

        How test will be performed: the user will upload the brain scans which will be read and processed then
        passed automatically by an automated script to the segmentation model. If the test is successful, the output
        of the test will be merged with the brain slice original image and visualized in the user interface.

\end{enumerate}

...

\subsection{Tests for Nonfunctional Requirements}

This sections presents the planned tests for the functional requirements for BrainInsight3D stated in SRS Document
\cite{BrainInsightSRS2024} in Section 5.2.

% \wss{The nonfunctional requirements for accuracy will likely just reference the
%   appropriate functional tests from above.  The test cases should mention
%   reporting the relative error for these tests.  Not all projects will
%   necessarily have nonfunctional requirements related to accuracy}

% \wss{Tests related to usability could include conducting a usability test and
%   survey.  The survey will be in the Appendix.}

% \wss{Static tests, review, inspections, and walkthroughs, will not follow the
%   format for the tests given below.}

\subsubsection{Accuracy Testing}

\paragraph{Title for Test}

\begin{enumerate}

  \item{test-id1\\}

        Type: Functional, Dynamic, Manual, Static etc.

        Initial State:

        Input/Condition:

        Output/Result:

        How test will be performed:

  \item{test-id2\\}

        Type: Functional, Dynamic, Manual, Static etc.

        Initial State:

        Input:

        Output:

        How test will be performed:

\end{enumerate}

\subsubsection{Usability Testing}
\item{test-id2\\}

Type: Functional, Dynamic, Manual, Static etc.

Initial State:

Input:

Output:

How test will be performed:

\end{enumerate}

\subsubsection{Maintability Testing}
\subsubsection{Portability Testing}

...

\subsection{Traceability Between Test Cases and Requirements}

\wss{Provide a table that shows which test cases are supporting which
  requirements.}

\section{Unit Test Description} \label{section6}

\wss{This section should not be filled in until after the MIS (detailed design
  document) has been completed.}

\wss{Reference your MIS (detailed design document) and explain your overall
  philosophy for test case selection.}

\wss{To save space and time, it may be an option to provide less detail in this section.
  For the unit tests you can potentially layout your testing strategy here.  That is, you
  can explain how tests will be selected for each module.  For instance, your test building
  approach could be test cases for each access program, including one test for normal behaviour
  and as many tests as needed for edge cases.  Rather than create the details of the input
  and output here, you could point to the unit testing code.  For this to work, you code
  needs to be well-documented, with meaningful names for all of the tests.}

\subsection{Unit Testing Scope}

\wss{What modules are outside of the scope.  If there are modules that are
  developed by someone else, then you would say here if you aren't planning on
  verifying them.  There may also be modules that are part of your software, but
  have a lower priority for verification than others.  If this is the case,
  explain your rationale for the ranking of module importance.}

\subsection{Tests for Functional Requirements}

\wss{Most of the verification will be through automated unit testing.  If
  appropriate specific modules can be verified by a non-testing based
  technique.  That can also be documented in this section.}

\subsubsection{Module 1}

\wss{Include a blurb here to explain why the subsections below cover the module.
  References to the MIS would be good.  You will want tests from a black box
  perspective and from a white box perspective.  Explain to the reader how the
  tests were selected.}

\begin{enumerate}

  \item{test-id1\\}

        Type: \wss{Functional, Dynamic, Manual, Automatic, Static etc. Most will
          be automatic}

        Initial State:

        Input:

        Output: \wss{The expected result for the given inputs}

        Test Case Derivation: \wss{Justify the expected value given in the Output field}

        How test will be performed:

  \item{test-id2\\}

        Type: \wss{Functional, Dynamic, Manual, Automatic, Static etc. Most will
          be automatic}

        Initial State:

        Input:

        Output: \wss{The expected result for the given inputs}

        Test Case Derivation: \wss{Justify the expected value given in the Output field}

        How test will be performed:

  \item{...\\}

\end{enumerate}

\subsubsection{Module 2}

...

\subsection{Tests for Nonfunctional Requirements}

\wss{If there is a module that needs to be independently assessed for
  performance, those test cases can go here.  In some projects, planning for
  nonfunctional tests of units will not be that relevant.}

\wss{These tests may involve collecting performance data from previously
  mentioned functional tests.}

\subsubsection{Module ?}

\begin{enumerate}

  \item{test-id1\\}

        Type: \wss{Functional, Dynamic, Manual, Automatic, Static etc. Most will
          be automatic}

        Initial State:

        Input/Condition:

        Output/Result:

        How test will be performed:

  \item{test-id2\\}

        Type: Functional, Dynamic, Manual, Static etc.

        Initial State:

        Input:

        Output:

        How test will be performed:

\end{enumerate}

\subsubsection{Module ?}

...

\subsection{Traceability Between Test Cases and Modules}

\wss{Provide evidence that all of the modules have been considered.}

\bibliographystyle{plainnat}

\bibliography{../../refs/ReferencesVnV}

\newpage

\section{Appendix}

This is where you can place additional information.

\subsection{Symbolic Parameters}

The definition of the test cases will call for SYMBOLIC\_CONSTANTS.
Their values are defined in this section for easy maintenance.

\subsection{Usability Survey Questions?}

\wss{This is a section that would be appropriate for some projects.}

\newpage{}
\section*{Appendix --- Reflection}

The information in this section will be used to evaluate the team members on the
graduate attribute of Lifelong Learning.  Please answer the following questions:

\newpage{}
\section*{Appendix --- Reflection}

\wss{This section is not required for CAS 741}

The information in this section will be used to evaluate the team members on the
graduate attribute of Lifelong Learning.  Please answer the following questions:

\begin{enumerate}
  \item What knowledge and skills will the team collectively need to acquire to
        successfully complete the verification and validation of your project?
        Examples of possible knowledge and skills include dynamic testing knowledge,
        static testing knowledge, specific tool usage etc.  You should look to
        identify at least one item for each team member.
  \item For each of the knowledge areas and skills identified in the previous
        question, what are at least two approaches to acquiring the knowledge or
        mastering the skill?  Of the identified approaches, which will each team
        member pursue, and why did they make this choice?
\end{enumerate}

\end{document}